%!TEX root = ../dokumentation.tex

\chapter{Einleitung}
Grundlage der Mathematik ist es, Sätze zu Beweisen. Wie jede von Menschenhand ausgeführte Tätigkeit, ist auch der von einem Mathematiker geführte Beweis fehleranfällig. Dies führte zur Entwicklung von automatischen Beweisverfahren, diese dienen der maschinellen Überprüfung eines Beweises. Der in dieser Arbeit erarbeitete Tableaubeweiser implementiert eines von vielen solcher Verfahren, welches insbesondere wegen der einfachen graphischen Nachvollziehbarkeit des Beweises einen guten Einstieg in das Thema bietet.

Der Beweiser ist in klassischer, sowie nichtklassischer Aussagen und Prädikatenlogik 1. Ordnung entwickelt. Dieser soll zur didaktischen Unterstützung von Studierenden dienen. Deshalb wurde neben dem automatischen Beweis, ein manueller Modus entwickelt, in dem der Studierende interaktiv eine Beweisführung durchführen kann und dabei Feedback über die Korrektheit der durchgeführten Schritte bekommt. Im automatischen Modus, soll dem Benutzer, nach Eingabe der Aussagen das berechnete Tableau präsentiert werden. Wichtig ist dabei, bei der Darstellung die durchgeführten Berechnungsschritte möglichst nachvollziehbar darzustellen. Der manuelle Modus hingegen, soll die Möglichkeit bieten, den jeweils nächsten Berechnungsschritt auszuwählen und das vom Benutzer eingegebene Ergebnis zu validieren. Hauptziel ist es dabei, dem Benutzer die Funktionsweise eines Tableau näher zu bringen. Ein dennoch sehr wichtiges Nebenziel davon ist aber auch, eine Berechnung von Tableaux, die im Automatik Modus nicht berechnet werden können durch den manuellen Modus berechenbar zu machen.

Hierfür gilt es zuerst ein grundsätzliches Verständnis der Logiken zu erarbeiten. Ist dieses geschaffen, muss darauf aufbauend die Frage, wie ein Tableau funktioniert und berechnet werden kann geklärt werden. Die Funktionsweise ist dabei für die verschiedenen Logiken zwar ähnlich bzw. aufeinander aufbauend aber dennoch muss die Frage für alle extra beantwortet werden. Eine weitere wichtige Fragestellung ist, wie das Tableau später dem Benutzer präsentiert werden kann. Dabei sind die eben vorgestellten Punkte einer einfachen Nachvollziehbarkeit und eine intuitive Interaktion im manuellen Modus kritische Punkte die es zu beachten gilt. Darauf aufbauend muss dann die Implementierung geplant und umgesetzt werden. Die genannten Punkte werden in den im folgenden kurz umrissenen Kapiteln erarbeitet.

Im Grundlagenkapitel wird zuerst eine grundsätzliche Einführung in das Thema Logik, sowie das Tableau-Beweisverfahren gegeben. Das Kapitel Automatische Beweisverfahren, liefert dann einen kurzen Blick über den Tellerrand und zeigt alternative Beweisverfahren, sowie einen Überblick über den Einsatz dieser auf. Anschließend wird im Kapitel Parsen und Datenstrukturen, der erste Schritt beim computergestützten Beweisverfahren behandelt, nämlich die Umwandlung von einer logischen Aussage als Zeichenkette zu einer Datenstruktur. Im nächsten Kapitel, Darstellung und Interaktion, werden verschiedene Darstellungs- und Bedienkonzepte erarbeitet, evaluiert und die konkrete Implementierung dieser behandelt. Zuletzt wird im Kapitel Implementierung, die Umsetzung des Herzstücks des Beweisers, in dem das Tableau berechnet wird behandelt.



